\documentclass{beamer}
\usepackage{hyperref}
\usepackage[CJKmath=true, AutoFakeBold = true]{xeCJK}
\usepackage[T1]{fontenc}
\setCJKmainfont{AR PL KaitiM GB} %中文字体设置为楷体。
% other packages
\usepackage{latexsym,xcolor,multicol,booktabs,calligra}
\usepackage{amssymb,amsfonts,amsmath,amsthm,mathrsfs,mathptmx} %数学常用宏包
\usepackage{graphicx,pstricks,listings,stackengine}
\usefonttheme[onlymath]{serif} %使用衬线字体展示公式,更加美观

% %origin
% \author{卢** \inst{1}、   张**\inst{1} }
% \title{江南大学学术论文汇报模板}
% \subtitle{Jiangnan University academic report beamer template}
% \institute[JNU]
% {\normalsize{\inst{1} 商学院\quad 江南大学\quad \textit{****@qq.com}}}
% % \date{xxxx年xx月xx日}
% \date{\today} %直接使用编译当天日期
% \usepackage{JNU}


\author{Junbo Dong}
\title{Title}
\subtitle{subtitle}
\institute[SUSTech]{dongjb2024@mail.sustech.edu.cn}
% {\normalsize{\inst{1} 商学院\quad 江南大学\quad \textit{****@qq.com}}}
% \date{xxxx年xx月xx日}
\date{\today} %直接使用编译当天日期
\usepackage{JNU}

% defs
\def\cmd#1{\texttt{\color[RGB]{0, 0, 139}\footnotesize $\backslash$#1}}
\def\env#1{\texttt{\color[RGB]{0, 0, 139}\footnotesize #1}}



\lstset{
    language=[LaTeX]TeX,
    basicstyle=\ttfamily\footnotesize,
    keywordstyle=\bfseries\color[RGB]{0, 0, 139},
    stringstyle=\color[RGB]{50, 50, 50},
    numbers=left,
    numberstyle=\small\color{gray},
    rulesepcolor=\color{red!20!green!20!blue!20},
    frame=shadowbox,
}

% \AtBeginSection[]{} % 取消章节切换前的目录
\begin{document}


\begin{frame}
    \titlepage
    % \begin{figure}[htpb]
    %     \begin{center}
    %         \includegraphics[width=0.55\linewidth]{pic/jiangnan_logo.png}
    %     \end{center}
    % \end{figure}
\end{frame}

\begin{frame}
    \tableofcontents[sectionstyle=show,subsectionstyle=show/shaded/hide,subsubsectionstyle=show/shaded/hide]
\end{frame}


\section{Why Beamer}

\begin{frame}{Why Beamer}
    \begin{itemize}
        \item \LaTeX 广泛用于学术界,期刊会议论文模板
    \end{itemize}
    \begin{table}[h]
        \centering
        \begin{tabular}{c|c}
            Microsoft\textsuperscript{\textregistered}  Word & \LaTeX \\
            \hline
            文字处理工具 & 专业排版软件 \\
            容易上手,简单直观 & 容易上手 \\
            所见即所得 & 所见即所想,所想即所得 \\
            高级功能不易掌握 & 进阶难,但一般用不到 \\
            处理长文档需要丰富经验 & 和短文档处理基本无异 \\
            花费大量时间调格式 & 无需担心格式,专心作者内容 \\
            公式排版差强人意 & 尤其擅长公式排版 \\
            二进制格式,兼容性差 & 文本文件,易读、稳定 \\
            付费商业许可 & 自由免费使用 \\
        \end{tabular}
    \end{table}
\end{frame}



\begin{frame}{依次展示列表}
    \begin{itemize}[<+-| alert@+>] % 当然,除了alert,手动在里面插 \pause 也行
        \item 大家都会\LaTeX{},好多学校都有自己的Beamer主题
        \item 中文支持请选择 Xe\LaTeX{} 编译选项
    \end{itemize}
\end{frame}

\begin{frame}[allowframebreaks]{列表自动溢出}
    \begin{itemize}
    \setlength{\itemsep}{10pt}
        \item 这是基于清华大佬的项目进行修改的模板
        \item 原模板地址 \url{https://www.overleaf.com/latex/templates/thu-beamer-theme/vwnqmzndvwyb}
        \item 我在原模板的基础上进行了一点修改
        \item 风格上更符合我个人的偏好
        \item 添加了一些小的Beamer使用示例
        \item 同原模板一样嫌弃默认的字体
        \item 所以更换成看起来更舒服的楷体
        \item 修正了原模板中文字体不完全统一的bug
        \item 将公式字体调整为衬线体,更加美观
        \item \alert{可以使用\cmd{alert}对文字进行强调}

    \end{itemize}
\end{frame}

\section{优势}
\subsection{A. }
\begin{frame}
    \begin{itemize}
        \item 有一些 \LaTeX{} 自带的
        \item \texttt{有一些Tsinghua的}
        \item 本模板来源自 \newline \url{https://www.latexstudio.net/archives/4051.html}
    \end{itemize}
\end{frame}

\subsection{B. }
\begin{frame}{插入图片}
    % \begin{figure}
    %     \centering
    %     \includegraphics[width=0.5\textwidth]{pic/jnu.jpg}
    %     \caption{学校logo}
    %     \label{fig:logo}
    % \end{figure}
\end{frame}

\section{原因}

\subsection{A. }

\begin{frame}{这一份主题与原始的THU Beamer Theme区别在于}
    \begin{itemize}
        \item 顶栏的小点变成一行而不是多行
        \item 中文采用楷书
    \end{itemize}
\end{frame}


\subsection{C. }
\begin{frame}{排版举例}
    \begin{exampleblock}{无编号公式} % 加 * 
        \begin{equation*}
            J(\theta) = \mathbb{E}_{\pi_\theta}[G_t] = \sum_{s\in\mathcal{S}} d^\pi (s)V^\pi(s)=\sum_{s\in\mathcal{S}} d^\pi(s)\sum_{a\in\mathcal{A}}\pi_\theta(a|s)Q^\pi(s,a)
        \end{equation*}
    \end{exampleblock}
    \begin{exampleblock}{多行多列公式\footnote{如果公式中有文字出现,请用 $\backslash$mathrm\{\} 或者 $\backslash$text\{\} 包含,不然就会变成 $clip$,在公式里看起来比 $\mathrm{clip}$ 丑非常多。}}
        % 使用 & 分隔
        \begin{align}
            公式Q_\mathrm{target}&=r+\gamma Q^\pi(s^\prime, \pi_\theta(s^\prime)+\epsilon)\\
            \epsilon&\sim\mathrm{clip}(\mathcal{N}(0, \sigma), -c, c)\nonumber
        \end{align}
    \end{exampleblock}
\end{frame}

\begin{frame}
    \begin{exampleblock}{编号多行公式}
        % Taken from Mathmode.tex
        \begin{multline}
            A=\lim_{n\rightarrow\infty}\Delta x\left(a^{2}+\left(a^{2}+2a\Delta x+\left(\Delta x\right)^{2}\right)\right.\label{eq:reset}\\
            +\left(a^{2}+2\cdot2a\Delta x+2^{2}\left(\Delta x\right)^{2}\right)\\
            +\left(a^{2}+2\cdot3a\Delta x+3^{2}\left(\Delta x\right)^{2}\right)\\
            +\ldots\\
            \left.+\left(a^{2}+2\cdot(n-1)a\Delta x+(n-1)^{2}\left(\Delta x\right)^{2}\right)\right)\\
            =\frac{1}{3}\left(b^{3}-a^{3}\right)
        \end{multline}
    \end{exampleblock}
\end{frame}

\begin{frame}{图形与分栏}
    % From thuthesis user guide.
    \begin{minipage}[c]{0.3\linewidth}
        \psset{unit=0.8cm}
        \begin{pspicture}(-1.75,-3)(3.25,4)
            \psline[linewidth=0.25pt](0,0)(0,4)
            \rput[tl]{0}(0.2,2){$\vec e_z$}
            \rput[tr]{0}(-0.9,1.4){$\vec e$}
            \rput[tl]{0}(2.8,-1.1){$\vec C_{ptm{ext}}$}
            \rput[br]{0}(-0.3,2.1){$\theta$}
            \rput{25}(0,0){%
            \psframe[fillstyle=solid,fillcolor=lightgray,linewidth=.8pt](-0.1,-3.2)(0.1,0)}
            \rput{25}(0,0){%
            \psellipse[fillstyle=solid,fillcolor=yellow,linewidth=3pt](0,0)(1.5,0.5)}
            \rput{25}(0,0){%
            \psframe[fillstyle=solid,fillcolor=lightgray,linewidth=.8pt](-0.1,0)(0.1,3.2)}
            \rput{25}(0,0){\psline[linecolor=red,linewidth=1.5pt]{->}(0,0)(0.,2)}
%           \psRotation{0}(0,3.5){$\dot\phi$}
%           \psRotation{25}(-1.2,2.6){$\dot\psi$}
            \psline[linecolor=red,linewidth=1.25pt]{->}(0,0)(0,2)
            \psline[linecolor=red,linewidth=1.25pt]{->}(0,0)(3,-1)
            \psline[linecolor=red,linewidth=1.25pt]{->}(0,0)(2.85,-0.95)
            \psarc{->}{2.1}{90}{112.5}
            \rput[bl](.1,.01){C}
        \end{pspicture}
    \end{minipage}\hspace{1cm}
    \begin{minipage}{0.5\linewidth}
        \medskip
        %\hspace{2cm}
        \begin{figure}[h]
            \centering
            \includegraphics[height=.4\textheight]{pic/dtmf.pdf}
        \end{figure}
    \end{minipage}
\end{frame}

\begin{frame}[fragile]{\LaTeX{} 常用命令}
    \begin{exampleblock}{命令}
        \centering
        \footnotesize
        \begin{tabular}{llll}
            \cmd{chapter} & \cmd{section} & \cmd{subsection} & \cmd{paragraph} \\
            章 & 节 & 小节 & 带题头段落 \\\hline
            \cmd{centering} & \cmd{emph} & \cmd{verb} & \cmd{url} \\
            居中对齐 & 强调 & 原样输出 & 超链接 \\\hline
            \cmd{footnote} & \cmd{item} & \cmd{caption} & \cmd{includegraphics} \\
            脚注 & 列表条目 & 标题 & 插入图片 \\\hline
            \cmd{label} & \cmd{cite} & \cmd{ref} \\
            标号 & 引用参考文献 & 引用图表公式等\\\hline
        \end{tabular}
    \end{exampleblock}
    \begin{exampleblock}{环境}
        \centering
        \footnotesize
        \begin{tabular}{lll}
            \env{table} & \env{figure} & \env{equation}\\
            表格 & 图片 & 公式 \\\hline
            \env{itemize} & \env{enumerate} & \env{description}\\
            无编号列表 & 编号列表 & 描述 \\\hline
        \end{tabular}
    \end{exampleblock}
\end{frame}

\begin{frame}[fragile]{\LaTeX{} 环境命令举例}
    \begin{minipage}{0.5\linewidth}
\begin{lstlisting}[language=TeX]
\begin{itemize}
  \item A \item B
  \item C
  \begin{itemize}
    \item C-1
  \end{itemize}
\end{itemize}
\end{lstlisting}
    \end{minipage}\hspace{1cm}
    \begin{minipage}{0.3\linewidth}
        \begin{itemize}
            \item A
            \item B
            \item C
            \begin{itemize}
                \item C-1
            \end{itemize}
        \end{itemize}
    \end{minipage}
    \medskip
    \pause
    \begin{minipage}{0.5\linewidth}
\begin{lstlisting}[language=TeX]
\begin{enumerate}
  \item 巨佬 \item 大佬
  \item 萌新
  \begin{itemize}
    \item[n+e] 瑟瑟发抖
  \end{itemize}
\end{enumerate}
\end{lstlisting}
    \end{minipage}\hspace{1cm}
    \begin{minipage}{0.3\linewidth}
        \begin{enumerate}
            \item 巨佬
            \item 大佬
            \item 萌新
            \begin{itemize}
                \item[n+e] 瑟瑟发抖
            \end{itemize}
        \end{enumerate}
    \end{minipage}
\end{frame}

\begin{frame}[fragile]{\LaTeX{} 数学公式}
    \begin{columns}
        \begin{column}{.55\textwidth}
\begin{lstlisting}[language=TeX]
$V = \frac{4}{3}\pi r^3$

\[
  V = \frac{4}{3}\pi r^3
\]

\begin{equation}
  \label{eq:vsphere}
  V = \frac{4}{3}\pi r^3
\end{equation}
\end{lstlisting}
        \end{column}
        \begin{column}{.4\textwidth}
            $V = \frac{4}{3}\pi r^3$
            \[
                V = \frac{4}{3}\pi r^3
            \]
            \begin{equation}
                \label{eq:vsphere}
                V = \frac{4}{3}\pi r^3
            \end{equation}
        \end{column}
    \end{columns}
    \begin{itemize}
        \item 更多内容请看 \href{https://zh.wikipedia.org/wiki/Help:数学公式}{\color{purple}{这里}}
    \end{itemize}
\end{frame}

\begin{frame}[fragile]
    \begin{columns}
        \column{.6\textwidth}
\begin{lstlisting}[language=TeX]
    \begin{table}[htbp]
      \caption{编号与含义}
      \label{tab:number}
      \centering
      \begin{tabular}{cl}
        \toprule
        编号 & 含义 \\
        \midrule
        1 & 4.0 \\
        2 & 3.7 \\
        \bottomrule
      \end{tabular}
    \end{table}
    公式~(\ref{eq:vsphere}) 的
    编号与含义请参见
    表~\ref{tab:number}。
\end{lstlisting}
        \column{.4\textwidth}
        \begin{table}[htpb]
            \centering
            \caption{编号与含义}
            \label{tab:number}
            \begin{tabular}{cl}\toprule
                编号 & 含义 \\\midrule
                1 & 4.0\\
                2 & 3.7\\\bottomrule
            \end{tabular}
        \end{table}
        \normalsize 公式~(\ref{eq:vsphere})的编号与含义请参见表~\ref{tab:number}。
    \end{columns}
\end{frame}

\begin{frame}{作图}
    \begin{itemize}
        \item 矢量图 eps, ps, pdf
        \begin{itemize}
            \item METAPOST, pstricks, pgf $\ldots$
            \item Xfig, Dia, Visio, Inkscape $\ldots$
            \item Matlab / Excel 等保存为 pdf
        \end{itemize}
        \item 标量图 png, jpg, tiff $\ldots$
        \begin{itemize}
            \item 提高清晰度,避免发虚
            \item 应尽量避免使用清晰度不够的图片
        \end{itemize}
    \end{itemize}

\end{frame}

\section{计划}
\begin{frame}
    \begin{itemize}
        \item 一月:完成文献调研
        \item 二月:复现并评测各种Beamer主题美观程度
        \item 三、四月:美化THU Beamer主题
        \item 五月:论文撰写
    \end{itemize}
\end{frame}


\section{参考文献}
\begin{frame}[allowframebreaks]
    \bibliography{ref}
    \bibliographystyle{alpha}
    % 如果参考文献太多的话,可以像下面这样调整字体:
    % \tiny\bibliographystyle{alpha}
\end{frame}

\begin{frame}
    \begin{center}
        {\Huge\calligra Thanks!}
    \end{center}
\end{frame}

\end{document}