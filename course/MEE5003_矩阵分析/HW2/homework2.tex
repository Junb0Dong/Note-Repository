% !TEX TS-program = xelatex
% !TEX encoding = UTF-8
% !Mode:: "TeX:UTF-8"

\documentclass[onecolumn,oneside]{SUSTechHomework}

\usepackage{blindtext}
\usepackage{enumerate}
% \usepackage{lastpage}
% \usepackage{fancyhdr}
% \pagestyle{fancy}
% \fancyfoot[C]{第 \thepage 页,共 \pageref{LastPage} 页}

\author{董骏博}
\sid{12432995}
\title{Homework 2}
\coursecode{MEE5003}
\coursename{矩阵分析与应用}

\begin{document}

  \maketitle
  \section{基与维数运算}
  \subsection{}
  \[
  \mathbf{S} = \text{span} \left\{
    \begin{pmatrix} 1 \\ 2 \\ -1 \\ 3 \end{pmatrix},
    \begin{pmatrix} 1 \\ 0 \\ 0 \\ 2 \end{pmatrix},
    \begin{pmatrix} 2 \\ 8 \\ -4 \\ 8 \end{pmatrix},
    \begin{pmatrix} 1 \\ 1 \\ 1 \\ 1 \end{pmatrix},
    \begin{pmatrix} 3 \\ 3 \\ 0 \\ 6 \end{pmatrix}
  \right\}
  \]
  由题可以得到
  \[
    4\begin{pmatrix} 1 \\ 2 \\ -1 \\ 3 \end{pmatrix} - 2\begin{pmatrix} 1 \\ 0 \\ 0 \\ 2 \end{pmatrix} = \begin{pmatrix} 2 \\ 8 \\ -4 \\ 8 \end{pmatrix}
  \]
  \[
    \begin{pmatrix} 1 \\ 2 \\ -1 \\ 3 \end{pmatrix} + \begin{pmatrix} 1 \\ 0 \\ 0 \\ 2 \end{pmatrix} + \begin{pmatrix} 1 \\ 1 \\ 1 \\ 1 \end{pmatrix} = \begin{pmatrix} 3 \\ 3 \\ 0 \\ 6 \end{pmatrix}
  \]
  并对其进行简化,可以得出\(S\)的维数为\(3\),基为
  \[ 
  \begin{pmatrix} 1 \\ 2 \\ -1 \\ 3 \end{pmatrix}, \begin{pmatrix} 1 \\ 0 \\ 0 \\ 2 \end{pmatrix}, \begin{pmatrix} 1 \\ 1 \\ 1 \\ 1 \end{pmatrix}
  \]

  \subsection{}
  \subsubsection{}
  由题目已知
  \[
    V_1 = \begin{pmatrix} 1 & -1 & 5 & -1 \\ 1 & 1 & -2 & 3\end{pmatrix}^T
  \]
  可以得出\(V_1\)的维数为2,基为
  \[
    \begin{pmatrix} 1 \\ -1 \\ 5 \\ -1\end{pmatrix},
    \begin{pmatrix} 1 \\ 1 \\ -2 \\ 3\end{pmatrix}
  \]
  因为
  \[
    V_2 = \begin{pmatrix}-3 & 4 & -2 \\ -1 & 3 & 1 \\ 1 & -1 & 1 \\ 0 & 1 & -1\end{pmatrix}
  \]
  经过化简,可以得出,\(V_2\)的维数为3,基为
  \[
    \alpha_1, \alpha_2, \alpha_3
  \]

  \subsubsection{}
  \[
    V_1 \cap V_2 = \begin{pmatrix}
     1 & 1 & -3 & 4 & -2 \\ -1 & 1 & -1 & 3 & 1 \\ 5 & -2 & 1 & -1 & 1 \\ -1 & 3 & 0 & 1 & -1
    \end{pmatrix}
  \]
 同1.1求解方式,可以求得\(V_1 \cap V_2 \)的子空间维数为4,基为
 \[
  \begin{pmatrix} 1 & 1 & -3 & 4 \\ -1 & 1 & -1 & 3 \\ 5 & -2 & 1 & -1 \\ -1 & 3 & 0 & 1 \end{pmatrix}
 \]
  \subsubsection{}
  \[\text{dim}(V_1 + V_2) = \text{dim}V_1 + \text{dim}V_2 - \text{dim}(V_1 \cap V_2) = 2 + 3 - 4 = 1 \]
  
  \section{}
  \subsection{}
  \subsubsection{}
  线性映射满足加法和数乘法
  加法:
  \[
    T(p_1(t) + p_2(t)) = T(p_1(t)) + T(p_2(t))
  \]
  数乘法:
  \[
    T(a \times p(t)) = a \times T(p(t))
  \]
  因此
  \[
  \mathbf{T} = \xi_k \mathbf{D}^k + \xi_{k-1} \mathbf{D}^{k-1} + \cdots + \xi_1 \mathbf{D} + \xi_0 \mathbf{I}
  \]
  是线性映射
  \subsubsection{}
  同理,加法:
  \[
    T(p_1(t) + p_2(t)) = t^np_1'(0) + t + t^np_2'(0) + t \neq  T(p_1(t)) +  T(p_2(t))
  \]
  因此,
  \[
  \mathbf{T}(p(t)) = t^n p'(0) + t
  \]不满足加法封闭性,不是线性映射

  \subsection{}
  由题可以得到
  \[
  \begin{bmatrix}
    \beta_1 & \beta_2 & \beta_3
  \end{bmatrix}
    =
  \begin{bmatrix}
    \alpha_1 & \alpha_2 & \alpha_3
  \end{bmatrix}
  \begin{bmatrix}
    1 & 1 & 1 \\ 0 & 1 & 1 \\ 0 & 0 & 1
  \end{bmatrix}
  \]
  可以求得过渡矩阵为
  \[
    P = \begin{bmatrix}
      1 & 1 & 1 \\ 0 & 1 & 1 \\ 0 & 0 & 1
    \end{bmatrix}, 
    P^{-1} = \begin{bmatrix}
      1 & -1 & 0 \\ 0 & 1 & -1 \\ 0 & 0 & 1
    \end{bmatrix}
  \]
  所以\(\mathcal{A}\)在基\(\beta\)下的表示矩阵为:
  \[
    PAP^{-1} = \begin{bmatrix}
      6 & 13 & 12 \\ -3 & -4 & -5 \\ 2 & 3 & 6
    \end{bmatrix}
  \]
  
  \subsection{}
  \subsubsection{}
  1. 绕原点逆时针旋转\(\frac{\pi}{4}\)时
  \[
    T_1 = \begin{bmatrix}
      \cos(\frac{\pi}{4}) & \cos(\frac{\pi}{4}) \\
      \sin(\frac{\pi}{4}) & \cos(\frac{\pi}{4})
    \end{bmatrix}
    = \begin{bmatrix}
      \frac{\sqrt{2}}{2} & -\frac{\sqrt{2}}{2} \\
      \frac{\sqrt{2}}{2} & \frac{\sqrt{2}}{2}
    \end{bmatrix}
  \]
  2. 保持\(y\)坐标不变,将\(y\)坐标的3倍加到\(x\)坐标上时
  \[
    T_2 = \begin{bmatrix}
      1 & 3 \\ 0 & 1
    \end{bmatrix}
  \]
  3. 镜像矩阵为
  \[
    T_3 = I - 2\frac{nn^T}{n^Tn} = \frac{1}{5}\begin{bmatrix}
      1 & -2 \\ -2 & -4
    \end{bmatrix}
  \]
  因此,可以得到
  \[
    T = T_3T_2T_1 = \frac{1}{5}\begin{bmatrix}
      \sqrt{2} & 0 \\ -6\sqrt{2} & -4\sqrt{2}
    \end{bmatrix}
  \]
  为线性变换

  \subsubsection{}
  \[
  \mathcal{B}(e_1) = Te_1 =     \frac{1}{5}\begin{bmatrix}
    1 & -2 \\ -2 & -4
  \end{bmatrix}
  \begin{bmatrix}
    1 & 3 \\ 0 & 1
  \end{bmatrix}
  \begin{bmatrix}
    \frac{\sqrt{2}}{2} & -\frac{\sqrt{2}}{2} \\
    \frac{\sqrt{2}}{2} & \frac{\sqrt{2}}{2}
  \end{bmatrix}
  \begin{bmatrix}
    1 \\ 0
  \end{bmatrix}
  = \begin{bmatrix}
    \frac{\sqrt{2}}{5} \\ -\frac{6\sqrt{2}}{5}
  \end{bmatrix}
  \]
  \[
    \mathcal{B}(e_2) = Te_2 =     \frac{1}{5}\begin{bmatrix}
      1 & -2 \\ -2 & -4
    \end{bmatrix}
    \begin{bmatrix}
      1 & 3 \\ 0 & 1
    \end{bmatrix}
    \begin{bmatrix}
      \frac{\sqrt{2}}{2} & -\frac{\sqrt{2}}{2} \\
      \frac{\sqrt{2}}{2} & \frac{\sqrt{2}}{2}
    \end{bmatrix}
    \begin{bmatrix}
      0 \\ 1
    \end{bmatrix}
    = \begin{bmatrix}
      0 \\ -\frac{4\sqrt{2}}{5}
    \end{bmatrix}
  \]
  \subsubsection{}
  \(\mathcal{B}\)的表示矩阵为
  \[  
  \frac{1}{5}\begin{bmatrix}
    \sqrt{2} & 0 \\ -6\sqrt{2} & -4\sqrt{2}
  \end{bmatrix}
  \]
  \[
    \mathcal{B}(a) =   \frac{1}{5}\begin{bmatrix}
      \sqrt{2} & 0 \\ -6\sqrt{2} & -4\sqrt{2}
    \end{bmatrix}
    \begin{bmatrix}
      3 \\ 5
    \end{bmatrix}
     = \begin{bmatrix}
      \frac{3\sqrt{2}}{5} \\ -\frac{38\sqrt{2}}{5}
     \end{bmatrix}
  \]

  \section{}
  \subsection{}
  \subsubsection{}
  通过\(det(A-\lambda I) = 0\)即可求出\(A\)的特征值为
  \[
  \lambda_1 = \lambda_2 = \lambda_3 = 2
  \]
  通过\((A-\lambda I)x = 0\)即可求出\(A\)的特征向量为
  \[
     \begin{bmatrix}
      0 & 1 & 1 \\ 0 & 2 & 2 \\ 1 & 0 & 1
     \end{bmatrix}
  \]
  因为\(A\)只存在两个线性无关的特征向量,所以\(A\)不能进行对角化

  \subsubsection{}
  同理。可以求出\(B\)的特征值为
  \[
     \lambda_1 = -1 , \lambda_2 = 2
  \]
  \(B\)的特征向量为
  \[
     \begin{bmatrix}
      1 & 1 & 1 \\ -1 & 0 & 1 \\ 0 & -1 & 1
     \end{bmatrix}
  \]
  特征向量的维数为3,所以\(B\)可以进行对角化
  \[
     B' = P^{-1}BP = \begin{bmatrix}
      -1 & 0 & 0 \\ 0 & -1 & 0 \\ 0 & 0 & 2
     \end{bmatrix}
  \]

  \subsubsection{}
  同理。可以求出\(C\)的特征值为
  \[
     \lambda_1 = 0 , \lambda_2 = i\sqrt{14}, \lambda_3 = -i\sqrt{14}
  \]
  \(C\)不能进行对角化
  % \(C\)的特征向量为
  % \[
  %    \begin{bmatrix}
  %     3 & \frac{2}{i\sqrt{14}} & \frac{1}{i\sqrt{14}} \\ -1 & 1 & 0\\ 2 & 0 & 1
  %    \end{bmatrix}
  % \]
  
  \subsection{}
  因为
  \[ AP_i = \lambda_i P_i \]
  \[ A^{-1}AP_i = A^{-1}\lambda_i P_i \] 
  \[ IP_i = A^{-1}\lambda_i P_i = \lambda_i A^{-1}P_i \]
  所以可以得到
  \[
     \frac{1}{\lambda_i} P_i = A^{-1}P_i
  \]
  所以,\(A^{-1}\)的特征值为\(\frac{1}{\lambda_i}\),\(A^{-1}\)的特征向量为\(P_i\)



\end{document}