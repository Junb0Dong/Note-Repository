% !TEX TS-program = xelatex
% !TEX encoding = UTF-8
% !Mode:: "TeX:UTF-8"

\documentclass[onecolumn,oneside]{SUSTechHomework}
\usepackage{blindtext}
\usepackage{enumerate}
 
\author{董骏博}
\sid{12432995}
\title{Homework 2}
\coursecode{MAE5009}
\coursename{Continuum Mechanics B}

\begin{document}
  \maketitle
  \subsection*{2-9}
  Since, 
   \[
   \epsilon_n = \frac{\epsilon_x + \epsilon_y}{2} + \frac{\epsilon_x - \epsilon_y}{2}  \cos2\alpha + \frac{\gamma_{xy}}{2} \sin 2\alpha
   \]
   
   \[
   \epsilon_t = \frac{\epsilon_x + \epsilon_y}{2} - \frac{\epsilon_x - \epsilon_y}{2}  \cos2\alpha - \frac{\gamma_{xy}}{2} \sin 2\alpha
   \]

   \[
   \gamma_{nt} = (\epsilon_y - \epsilon_x) \sin2\alpha + \gamma_{xy}\cos2\alpha
   \]

   where\(\alpha = 22.5^{\circ}\)

   Therefore, we can get 
   \[\epsilon_n = 0.001\]
   \[\epsilon_t = -0.001\]
   \[\gamma_{nt} = 0.002\]

  \subsection*{2-10}

  \[
    \epsilon_{ab} = \epsilon_{x' y'} = \frac{\epsilon_x + \epsilon_y}{2} + \frac{\epsilon_x - \epsilon_y}{2}  \cos2\alpha + \frac{\gamma_{xy}}{2} \sin 2\alpha
  \]
  where, \[\alpha = arctan\frac{3}{4} = 36.87^{\circ}\]
  Therefore, we can get \[\epsilon_{ab} = 0.004296\]
  Since, \[\epsilon_{ab} = \frac{\delta ab}{ab}\]
  Therefore, \[\delta ab = 0.02148'' \]

  \subsection*{2-6}
  Since, \[\tan2\alpha = \frac{\gamma_{xy}}{\epsilon_x - \epsilon_y}\]
  When \(\alpha\) is 45°(\(\gamma_{xy} = \epsilon_x - \epsilon_y\)), the shear stress is maximum, the direction of the maximum shear strain can be determined by the following formula
  \[\gamma_{max} = \sqrt{(\frac{\epsilon_x - \epsilon_y}{2})^2 + (\frac{\gamma_{xy}}{2})^2}\]

  \subsection*{2-7}
  \begin{enumerate}[(a)]
    \item Since, 
    \[\frac{\delta u}{\delta x} = 2kx, \frac{\delta v}{\delta x} = 2kx, \frac{\delta u}{\delta y} = 2ky, \frac{\delta v}{\delta y} = -6ky\]
    we can get 
    \[A' B' = D' C'\sqrt{(dx+\frac{\delta u}{\delta x}dx)^2 + (\frac{\delta v}{\delta x}dx)^2} = dx\sqrt{20k^2 + 4k + 1}\]
    \[A' D' = B' C'\sqrt{(dy+\frac{\delta v}{\delta y}dy)^2 + (\frac{\delta u}{\delta y}dy)^2} = dy\sqrt{40k^2 - 12k + 1}\]
    The angular positions of \(A' B'\) and \(A' D'\) is
    \[
    \theta = \tan \theta = \frac{\delta v}{\delta x} = 2kx = 4\times 10^{-4}
    \]
    \[
    -\lambda = -\tan \lambda = \frac{\delta u}{\delta y} = 2ky = 2\times 10^{-4}
    \]
    % substituting x and y of point A into \[u = k(2x+y^2), v = k(x^2-3y^2)\]
    
    % we get \[u_A = 5\times10^{-4}, v_A = 1\times 10^{-4}\]
    % Since \(A\) is \((2, 1, 0)\)

    \item The coordinates of point A after the displacement is \[x' = x + u_A = 2.0005\] \[y' = 1 + v_A = 1.0001\]
    So, the coordinates of point \(A' = (2.0005, 1.0001, 0)\) 
    \item \[w_z = \frac{1}{2}(\frac{\delta v}{\delta x} - \frac{\delta u}{\delta y}) = \frac{1}{2}(2kx-2ky) = 1\times 10^{-4}\]
  \end{enumerate}

\end{document}